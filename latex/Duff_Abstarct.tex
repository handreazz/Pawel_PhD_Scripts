\documentclass[11pt,letterpaper]{article}
\usepackage[utf8]{inputenc}
\usepackage{graphicx}
\usepackage{amsmath}
\usepackage{amsfonts}
\usepackage{amssymb}
\usepackage{float} %necessary to make [H] work for figures (place in text)
%\usepackage[sorting=none]{biblatex}
%\bibliography{library}
\usepackage[superscript,biblabel]{cite}
\title{(yeaheyah)}
\author{Pawe\l{} Janowski}
\setlength{\topmargin}{0in}
\setlength{\headheight}{0in}
\setlength{\headsep}{0in}
\setlength{\textheight}{9in}
\setlength{\oddsidemargin}{0in}
\setlength{\textwidth}{6.5in}
%\usepackage{fullpage}
\DeclareUnicodeCharacter{00A0}{ }
\parskip 8pt
\begin{document}
    \begin{flushright}
    \parskip 0pt
    Pawe\l{} Janowski\\
    Spring 2014 Duff Travel Award\\
    \today
    \vspace{10 mm}
    \end{flushright}
    
    



\section*{Title: Implementation of Amber Molecular Dynamics Force Field for Improved Crystallographic Refinement in Phenix}
\subsection*{Authors: Pawe\l{} Janowski, Nigel Moriarty, Paul Adams, David A. Case }
\subsection*{Affiliations: Rutgers Univeristy Dept. of Chemistry and Chemical Biology, Lawrence Berkeley National Laboratory Div. of Phys. Sci.}

Crystallography has played a defining role in the advancement of biological chemistry. Within this field structure refinement is pivotal for obtaining accurate structures of macromolecules that both agree well with experimental data and are chemically sensible. Traditionally, crystallographic refinement has employed a semi-primitive set of heuristic geometry restraints. Here we present Phenix.Amber, a new approach to crystal refinement that significantly improves on results obtained with traditional restraints. By combining the Amber molecular dynamics force field with the Phenix suite for automated crystallograpy, we introduce a sophisticated set of restraints that not only account for bond, angle and dihedral energy terms but also van der Waals interactions and short and long range electrostatics. We show that Amber-based refinement leads to improved R-factors while at the same time significantly improving the structural characteristics of macromolecules as validated by accepted standards. In particular, our approach yields optimal results where electrostatics play a key structural role, such as in nucleic acids and structures where hydrogen bonding is crucial. From a technical standpoint phenix.Amber is a fully integrated computational that  ensures maximum computational efficiency. Our package provides full support for all crystallographic space groups and special residues or ligands. A user-friendly tool is provided to automate and simplify the set-up necessary to run both Amber and Phenix. Our approach will lead to improved macromolecular structure refinement through a more accurate interpretation of experimental diffraction data. 



\end{document}
