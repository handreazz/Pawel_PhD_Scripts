\documentclass[11pt,letterpaper]{article}
\usepackage[utf8]{inputenc}
\usepackage{graphicx}
\usepackage{amsmath}
\usepackage{amsfonts}
\usepackage{amssymb}
\usepackage{float} %necessary to make [H] work for figures (place in text)
\usepackage{subfig}
\title{(yeaheyah)}
\author{Pawe\l{} Janowski}
\setlength{\topmargin}{0in}
\setlength{\headheight}{0in}
\setlength{\headsep}{0in}
\setlength{\textheight}{9in}
\setlength{\oddsidemargin}{0in}
\setlength{\textwidth}{6.5in}
%\usepackage{fullpage}
\parskip 8pt
\usepackage{Sweave}
\begin{document}
\setkeys{Gin}{width=0.5\textwidth}
\begin{flushright}
\parskip 0pt
Pawe\l{} Janowski
 
\today
\vspace{10 mm}
\end{flushright}
\begin{center}
\begin{Large}
\textbf{Applied Basic Statistics
Final Exam: Take Home Part}
\vspace{10 mm}
\end{Large}
\end{center}


\begin{enumerate}
\item Problem 1 \\
a.

%<<echo=FALSE>>=
\begin{Schunk}
\begin{Sinput}
> library("faraway", lib.loc = "/home/pjanowsk/Rlib")
> library(lattice)
> boxplot(pulp$bright ~ pulp$operator, horizontal = TRUE, xlab = "brightness", 
+     ylab = "operator")
\end{Sinput}
\end{Schunk}
\includegraphics{final-001}

b.
\begin{Schunk}
\begin{Sinput}
> model = aov(bright ~ operator, data = pulp)
> summary(model)
\end{Sinput}
\begin{Soutput}
            Df Sum Sq Mean Sq F value  Pr(>F)  
operator     3   1.34 0.44667  4.2039 0.02261 *
Residuals   16   1.70 0.10625                  
---
Signif. codes:  0 ‘***’ 0.001 ‘**’ 0.01 ‘*’ 0.05 ‘.’ 0.1 ‘ ’ 1 
\end{Soutput}
\end{Schunk}

The p-value of the anova F-test is 0.023 so at significance level 0.05 there is enough evidence to conclude that shift operators do affect pulp brightness.\\

c. \\
The samples have normal distributions (approximately):

\begin{Schunk}
\begin{Sinput}
> print(qqmath(~bright | operator, data = pulp, prepanel = prepanel.qqmathline, 
+     panel = function(x, ...) {
+         panel.qqmathline(x, ...)
+         panel.qqmath(x, ...)
+     }))
\end{Sinput}
\end{Schunk}
\includegraphics{final-003}

And the normal residual plot indicates that variances are equal:

\begin{Schunk}
\begin{Sinput}
> print(qqmath(~residuals, data = model, prepanel = prepanel.qqmathline, 
+     panel = function(x, ...) {
+         panel.qqmathline(x, ...)
+         panel.qqmath(x, ...)
+     }))
\end{Sinput}
\end{Schunk}
\includegraphics{final-004}

d.
\begin{Schunk}
\begin{Sinput}
> TukeyHSD(model)
\end{Sinput}
\begin{Soutput}
  Tukey multiple comparisons of means
    95% family-wise confidence level

Fit: aov(formula = bright ~ operator, data = pulp)

$operator
     diff         lwr       upr     p adj
b-a -0.18 -0.76981435 0.4098143 0.8185430
c-a  0.38 -0.20981435 0.9698143 0.2903038
d-a  0.44 -0.14981435 1.0298143 0.1844794
c-b  0.56 -0.02981435 1.1498143 0.0657945
d-b  0.62  0.03018565 1.2098143 0.0376691
d-c  0.06 -0.52981435 0.6498143 0.9910783
\end{Soutput}
\end{Schunk}

Based on these results I would say that the evidence is very strong that operators a and b have similar means and c and d have similar means. At 95\% CI a has the same mean as c and d but this is borderline. The evidence strongly suggests that the means of b and d are different.

\item Problem 2
a.
\begin{Schunk}
\begin{Sinput}
> str(barley)
\end{Sinput}
\begin{Soutput}
'data.frame':	120 obs. of  4 variables:
 $ yield  : num  27 48.9 27.4 39.9 33 ...
 $ variety: Factor w/ 10 levels "Svansota","No. 462",..: 3 3 3 3 3 3 7 7 7 7 ...
 $ year   : Factor w/ 2 levels "1932","1931": 2 2 2 2 2 2 2 2 2 2 ...
 $ site   : Factor w/ 6 levels "Grand Rapids",..: 3 6 4 5 1 2 3 6 4 5 ...
\end{Soutput}
\begin{Sinput}
> table(barley$variety)
\end{Sinput}
\begin{Soutput}
        Svansota          No. 462        Manchuria          No. 475 
              12               12               12               12 
          Velvet         Peatland          Glabron          No. 457 
              12               12               12               12 
Wisconsin No. 38            Trebi 
              12               12 
\end{Soutput}
\end{Schunk}

b.
\begin{Schunk}
\begin{Sinput}
> x = barley[barley$year == 1931, ]
> model2 = aov(yield ~ variety + site, data = x)
> summary(model2)
\end{Sinput}
\begin{Soutput}
            Df Sum Sq Mean Sq F value    Pr(>F)    
variety      9  646.3   71.81  3.6799  0.001612 ** 
site         5 5142.3 1028.45 52.7061 < 2.2e-16 ***
Residuals   45  878.1   19.51                      
---
Signif. codes:  0 ‘***’ 0.001 ‘**’ 0.01 ‘*’ 0.05 ‘.’ 0.1 ‘ ’ 1 
\end{Soutput}
\end{Schunk}

Yes, even at a 0.01 confidence level, both location and variety have an effect on yield.

c. 
\begin{Schunk}
\begin{Sinput}
> TukeyHSD(model2, ordered = TRUE, conf.level = 0.99)
\end{Sinput}
\begin{Soutput}
  Tukey multiple comparisons of means
    99% family-wise confidence level
    factor levels have been ordered

Fit: aov(formula = yield ~ variety + site, data = x)

$variety
                                 diff       lwr      upr     p adj
Svansota-No. 475            2.1944450 -7.809974 12.19886 0.9969345
Manchuria-No. 475           2.3777800 -7.626639 12.38220 0.9944400
Velvet-No. 475              2.6722200 -7.332199 12.67664 0.9872452
Peatland-No. 475            4.7666667 -5.237753 14.77109 0.6887850
Glabron-No. 475             5.5111100 -4.493309 15.51553 0.4969561
No. 462-No. 475             7.2388950 -2.765524 17.24331 0.1546470
No. 457-No. 475             8.4333350 -1.571084 18.43775 0.0525890
Wisconsin No. 38-No. 475    8.7666683 -1.237751 18.77109 0.0377288
Trebi-No. 475              10.6499933  0.645574 20.65441 0.0047635
Manchuria-Svansota          0.1833350 -9.821084 10.18775 1.0000000
Velvet-Svansota             0.4777750 -9.526644 10.48219 1.0000000
Peatland-Svansota           2.5722217 -7.432198 12.57664 0.9902259
Glabron-Svansota            3.3166650 -6.687754 13.32108 0.9484011
No. 462-Svansota            5.0444500 -4.959969 15.04887 0.6180347
No. 457-Svansota            6.2388900 -3.765529 16.24331 0.3243819
Wisconsin No. 38-Svansota   6.5722233 -3.432196 16.57664 0.2581796
Trebi-Svansota              8.4555483 -1.548871 18.45997 0.0514577
Velvet-Manchuria            0.2944400 -9.709979 10.29886 1.0000000
Peatland-Manchuria          2.3888867 -7.615533 12.39331 0.9942477
Glabron-Manchuria           3.1333300 -6.871089 13.13775 0.9635638
No. 462-Manchuria           4.8611150 -5.143304 14.86553 0.6650620
No. 457-Manchuria           6.0555550 -3.948864 16.05997 0.3646416
Wisconsin No. 38-Manchuria  6.3888883 -3.615531 16.39331 0.2934298
Trebi-Manchuria             8.2722133 -1.732206 18.27663 0.0614706
Peatland-Velvet             2.0944467 -7.909973 12.09887 0.9978488
Glabron-Velvet              2.8388900 -7.165529 12.84331 0.9807585
No. 462-Velvet              4.5666750 -5.437744 14.57109 0.7373311
No. 457-Velvet              5.7611150 -4.243304 15.76553 0.4342116
Wisconsin No. 38-Velvet     6.0944483 -3.909971 16.09887 0.3558883
Trebi-Velvet                7.9777733 -2.026646 17.98219 0.0811033
Glabron-Peatland            0.7444433 -9.259976 10.74886 0.9999996
No. 462-Peatland            2.4722283 -7.532191 12.47665 0.9926281
No. 457-Peatland            3.6666683 -6.337751 13.67109 0.9083807
Wisconsin No. 38-Peatland   4.0000017 -6.004418 14.00442 0.8558989
Trebi-Peatland              5.8833267 -4.121093 15.88775 0.4046719
No. 462-Glabron             1.7277850 -8.276634 11.73220 0.9995268
No. 457-Glabron             2.9222250 -7.082194 12.92664 0.9766974
Wisconsin No. 38-Glabron    3.2555583 -6.748861 13.25998 0.9538720
Trebi-Glabron               5.1388833 -4.865536 15.14330 0.5935032
No. 457-No. 462             1.1944400 -8.809979 11.19886 0.9999781
Wisconsin No. 38-No. 462    1.5277733 -8.476646 11.53219 0.9998267
Trebi-No. 462               3.4110983 -6.593321 13.41552 0.9390866
Wisconsin No. 38-No. 457    0.3333333 -9.671086 10.33775 1.0000000
Trebi-No. 457               2.2166583 -7.787761 12.22108 0.9966934
Trebi-Wisconsin No. 38      1.8833250 -8.121094 11.88774 0.9990586

$site
                                  diff        lwr       upr     p adj
Morris-Grand Rapids           0.233334 -6.8530524  7.319720 0.9999965
Duluth-Grand Rapids           1.239998 -5.8463884  8.326384 0.9883425
University Farm-Grand Rapids  6.773331 -0.3130554 13.859717 0.0155063
Crookston-Grand Rapids       14.606664  7.5202776 21.693050 0.0000000
Waseca-Grand Rapids          25.293331 18.2069446 32.379717 0.0000000
Duluth-Morris                 1.006664 -6.0797224  8.093050 0.9955462
University Farm-Morris        6.539997 -0.5463894 13.626383 0.0213083
Crookston-Morris             14.373330  7.2869436 21.459716 0.0000001
Waseca-Morris                25.059997 17.9736106 32.146383 0.0000000
University Farm-Duluth        5.533333 -1.5530534 12.619719 0.0757032
Crookston-Duluth             13.366666  6.2802796 20.453052 0.0000003
Waseca-Duluth                24.053333 16.9669466 31.139719 0.0000000
Crookston-University Farm     7.833333  0.7469466 14.919719 0.0033408
Waseca-University Farm       18.520000 11.4336136 25.606386 0.0000000
Waseca-Crookston             10.686667  3.6002806 17.773053 0.0000330
\end{Soutput}
\end{Schunk}

Strictly speaking, at confidence level 0.01, there is no significant difference in yields from the various crop varieties. However, knowing how Tukey's method should be loosely interpreted, I would summarize that Svansota, Manchuria, Velvet, 475 produce similar, lower yields; Trebi, Wisconsin, 457, 462 produce similar higher yields; and Peatlan and Glabron are about the same.

On the other hand the differences in yield due to location are very significant. Morris, Grand Rapids, Duluth all produce similar yields. U. Farm is also not statistically different from those three. The other locations are all very different from this group of 3-4 and from each other.

\end{enumerate}
\end{document}
