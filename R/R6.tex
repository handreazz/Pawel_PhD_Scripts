\documentclass[11pt,letterpaper]{article}
\usepackage[utf8]{inputenc}
\usepackage{graphicx}
\usepackage{amsmath}
\usepackage{amsfonts}
\usepackage{amssymb}
\usepackage{float} %necessary to make [H] work for figures (place in text)
\usepackage{subfig}
\title{(yeaheyah)}
\author{Pawe\l{} Janowski}
\setlength{\topmargin}{0in}
\setlength{\headheight}{0in}
\setlength{\headsep}{0in}
\setlength{\textheight}{9in}
\setlength{\oddsidemargin}{0in}
\setlength{\textwidth}{6.5in}
%\usepackage{fullpage}
\parskip 8pt
\usepackage{Sweave}
\begin{document}
\setkeys{Gin}{width=0.5\textwidth}
\begin{flushright}
\parskip 0pt
Pawe\l{} Janowski
 
\today
\vspace{10 mm}
\end{flushright}
\begin{center}
\begin{Large}
\textbf{Basic Applied Statistics
Assignment 6: Two-sample hypothesis testing}
\vspace{10 mm}
\end{Large}
\end{center}


\begin{enumerate}
\item Problem 23 \\
a.

%<<echo=FALSE>>=
%<<fig=TRUE,width=4,height=3>>=
%library("Devore7", lib.loc="/home/pjanowsk/Rlib")
%library(lattice)
%print(qqmath(~strength | type, data = xmp09.07, prepanel = prepanel.qqmathline,panel = function(x, ...) {
%panel.qqmathline(x, ...)
%panel.qqmath(x, ...)
%}))
%print(bwplot(type ~ strength, data = xmp09.07))
%@
\begin{Schunk}
\begin{Sinput}
> d = ex09.23
> qqnorm(d$H)
> qqline(d$H)
\end{Sinput}
\end{Schunk}
\includegraphics{R6-001}

b.
\begin{Schunk}
\begin{Sinput}
> boxplot(d, horizontal = TRUE)
\end{Sinput}
\end{Schunk}
\includegraphics{R6-002}

The boxplot does not suggest that there is much difference between the true average extensibilities.

c. The true average extensibility does not differ for the two types of fabric:

\begin{Schunk}
\begin{Sinput}
> t.test(d$H, d$P, alt = "two.sided")
\end{Sinput}
\begin{Soutput}
	Welch Two Sample t-test

data:  d$H and d$P 
t = -0.3801, df = 10.482, p-value = 0.7115
alternative hypothesis: true difference in means is not equal to 0 
95 percent confidence interval:
 -0.5403506  0.3820172 
sample estimates:
mean of x mean of y 
 1.508333  1.587500 
\end{Soutput}
\end{Schunk}

\item Problem 31\\
a. The boxplots seem to indicate that there is no difference between the true means. 
\begin{Schunk}
\begin{Sinput}
> e = ex09.31
> d = ex07.33
> nm <- list(d, e)
> ed <- sapply(mylist, "[", seq(max(sapply(mylist, length))))
> boxplot(ed, horizontal = TRUE)
\end{Sinput}
\end{Schunk}
\includegraphics{R6-004}

b. At 95 \% confidence level the null hypothesis cannot be rejected. There is no difference in the means.
\begin{Schunk}
\begin{Sinput}
> t.test(d, e, alt = "two.sided")
\end{Sinput}
\begin{Soutput}
	Welch Two Sample t-test

data:  d and e 
t = 0.1994, df = 23.869, p-value = 0.8437
alternative hypothesis: true difference in means is not equal to 0 
95 percent confidence interval:
 -7.854361  9.533506 
sample estimates:
mean of x mean of y 
 438.2941  437.4545 
\end{Soutput}
\end{Schunk}

\item Problem 36\\
\begin{Schunk}
\begin{Sinput}
> d = ex09.36
> t.test(d$U, d$A, paired = TRUE, conf.level = 0.99)
\end{Sinput}
\begin{Soutput}
	Paired t-test

data:  d$U and d$A 
t = 1.7286, df = 7, p-value = 0.1275
alternative hypothesis: true difference in means is not equal to 0 
99 percent confidence interval:
 -7.427283 21.927283 
sample estimates:
mean of the differences 
                   7.25 
\end{Soutput}
\end{Schunk}
At the 99 \% confidence level the null hypothesis cannot be rejected.

\item Problem 38\\
a. The means are really far apart. It seems that there is a big difference between the two types of concrete.
\begin{Schunk}
\begin{Sinput}
> d = ex09.38
> boxplot(d$Normal, d$High)
\end{Sinput}
\end{Schunk}
\includegraphics{R6-007}

b.An estimate of the difference between the means is:
\begin{Schunk}
\begin{Sinput}
> abs(mean(d$Normal) - mean(d$High))
\end{Sinput}
\begin{Soutput}
[1] 42.23333
\end{Soutput}
\end{Schunk}

A t-test at a 99\% confidence level indicates that the means are not identical:
\begin{Schunk}
\begin{Sinput}
> t.test(d$Normal, d$High, paired = TRUE, conf.level = 0.99)
\end{Sinput}
\begin{Soutput}
	Paired t-test

data:  d$Normal and d$High 
t = -37.7177, df = 14, p-value = 1.756e-15
alternative hypothesis: true difference in means is not equal to 0 
99 percent confidence interval:
 -45.56657 -38.90010 
sample estimates:
mean of the differences 
              -42.23333 
\end{Soutput}
\end{Schunk}

\item Problem 53\\
a.
\begin{Schunk}
\begin{Sinput}
> x = c(0.176 * 529, 0.158 * 563)
> n = c(529, 563)
> prop.test(x, n, alt = "two.sided", correct = FALSE)
\end{Sinput}
\begin{Soutput}
	2-sample test for equality of proportions without continuity
	correction

data:  x out of n 
X-squared = 0.6361, df = 1, p-value = 0.4251
alternative hypothesis: two.sided 
95 percent confidence interval:
 -0.02628156  0.06228156 
sample estimates:
prop 1 prop 2 
 0.176  0.158 
\end{Soutput}
\end{Schunk}
The p-value is high and indicates that at the 5\% significance level there is not enough evidence to suggest rejecting the hypothesis that the proportions do not differ.\\
b.
\begin{Schunk}
\begin{Sinput}
> power.prop.test(p1 = 0.15, p2 = 0.2, sig.level = 0.05, power = 0.9, 
+     alternative = "two.sided")
\end{Sinput}
\begin{Soutput}
     Two-sample comparison of proportions power calculation 

              n = 1211.529
             p1 = 0.15
             p2 = 0.2
      sig.level = 0.05
          power = 0.9
    alternative = two.sided

 NOTE: n is number in *each* group 
\end{Soutput}
\end{Schunk}
So you would need at least 1212 measurement in each sample.\\

\item Problem 2
a.
\begin{Schunk}
\begin{Sinput}
> 64.9 - 63.1
\end{Sinput}
\begin{Soutput}
[1] 1.8
\end{Soutput}
\begin{Sinput}
> qnorm(0.95) * sqrt((0.09^2) + (0.11^2))
\end{Sinput}
\begin{Soutput}
[1] 0.2337776
\end{Soutput}
\end{Schunk}
%So the 95\% CI is $1.8\pm .23$\\
b.\\
$\Delta_0=1$\\
z=
\begin{Schunk}
\begin{Sinput}
> (64.9 - 63.1 - 1)/sqrt((0.09^2) + (0.11^2))
\end{Sinput}
\begin{Soutput}
[1] 5.62878
\end{Soutput}
\end{Schunk}
z=5.63
\begin{Schunk}
\begin{Sinput}
> qnorm(0.999)
\end{Sinput}
\begin{Soutput}
[1] 3.090232
\end{Soutput}
\end{Schunk}
z is much greater than the cutoff so the hypothesis is rejected.\\

c. The p-value is:
\begin{Schunk}
\begin{Sinput}
> 1 - pnorm(169.38)
\end{Sinput}
\begin{Soutput}
[1] 0
\end{Soutput}
\end{Schunk}
This is ultra low and I would reject the null hypothesis at any confidence level in favor of the hypothesis that the true difference between the means is greater than 1.\\

d.$H_0:\mu_2-\mu_1=1$ and $H_a:\mu_2-\mu_1>1$

\item Problem 5

a. $H_a$ says that the mean heat output for non-sick people is more than 1cal/cm/min higher than the average heat output for sick people.
\begin{Schunk}
\begin{Sinput}
> (0.64 - 2.05 + 1)/sqrt((0.2^2/10) + (0.4^2/10))
\end{Sinput}
\begin{Soutput}
[1] -2.899138
\end{Soutput}
\begin{Sinput}
> -qnorm(0.99)
\end{Sinput}
\begin{Soutput}
[1] -2.326348
\end{Soutput}
\end{Schunk}
$H_0$ is rejected.\\

b.
\begin{Schunk}
\begin{Sinput}
> pnorm((0.64 - 2.05 + 1)/sqrt((0.2^2/10) + (0.4^2/10)))
\end{Sinput}
\begin{Soutput}
[1] 0.001870952
\end{Soutput}
\end{Schunk}

c.
\begin{Schunk}
\begin{Sinput}
> 1 - pnorm(-qnorm(0.99) - ((-1.2 - -1)/sqrt((0.2^2/10) + (0.4^2/10))))
\end{Sinput}
\begin{Soutput}
[1] 0.819151
\end{Soutput}
\end{Schunk}
I don't understand why the following does not work:
\begin{Schunk}
\begin{Sinput}
> power.t.test(n = 20, delta = 0.2, sd = 0.1414, type = "two.sample", 
+     alt = "one.sided")
\end{Sinput}
\begin{Soutput}
     Two-sample t test power calculation 

              n = 20
          delta = 0.2
             sd = 0.1414
      sig.level = 0.05
          power = 0.9969881
    alternative = one.sided

 NOTE: n is number in *each* group 
\end{Soutput}
\end{Schunk}

d.
\begin{Schunk}
\begin{Sinput}
> (0.4^2 + 0.2^2) * (qnorm(0.99) + qnorm(0.82))^2/(-1.2 - -1)^2
\end{Sinput}
\begin{Soutput}
[1] 52.54351
\end{Soutput}
\end{Schunk}
So use n=53.

\item Problem 11
\begin{Schunk}
\begin{Sinput}
> m = 5.5 - 3.8
> n = qnorm(0.975) * sqrt((0.3^2) + (0.2^2))
> m - n
\end{Sinput}
\begin{Soutput}
[1] 0.9933249
\end{Soutput}
\begin{Sinput}
> m + n
\end{Sinput}
\begin{Soutput}
[1] 2.406675
\end{Soutput}
\end{Schunk}
With 95\% confidence, the true difference between the means is in [.99,2.41].

\item Problem 24
a.
$\overline{x}=13.4$
\begin{Schunk}
\begin{Sinput}
> qt(0.95, 64) * 2.5
\end{Sinput}
\begin{Soutput}
[1] 4.172533
\end{Soutput}
\end{Schunk}
The upper bound is 13.4+4.2=17.6

b.
\begin{Schunk}
\begin{Sinput}
> (13.4 - 9.7)/sqrt((2.05^2 + 1.76^2))
\end{Sinput}
\begin{Soutput}
[1] 1.369422
\end{Soutput}
\begin{Sinput}
> f = (2.05^2 + 1.76^2)/((2.05^4/64) + (1.76^4/49))
> qt(0.95, f)
\end{Sinput}
\begin{Soutput}
[1] 1.749532
\end{Soutput}
\end{Schunk}
Fail to reject the hypothesis because 1.37<1.75. So there doesn't seem to be a difference.\\

c.
\begin{Schunk}
\begin{Sinput}
> m = 38.4 - 9.7
> f = (5.06^2 + 1.76^2)/((5.06^4/45) + (1.76^4/49))
> n = qt(0.975, f) * sqrt(5.06^2 + 1.76^2)
> m - n
\end{Sinput}
\begin{Soutput}
[1] 5.001855
\end{Soutput}
\begin{Sinput}
> m + n
\end{Sinput}
\begin{Soutput}
[1] 52.39815
\end{Soutput}
\end{Schunk}
With 95\% confidence the difference between the true average times is in the interval [5,52].

\item Problem 26
\begin{Schunk}
\begin{Sinput}
> 1 - pt(3.63, 37.5)
\end{Sinput}
\begin{Soutput}
[1] 0.0004209356
\end{Soutput}
\end{Schunk}
So P-value =.0004 which is less than .01 so the null hypothesis is rejected and we can say that the true average drop is higher for alloy connections.\\

\item Problem 34
a. $\overline{x}-\overline{y}\pm t_{\alpha,\nu}*S_p\sqrt{\frac{1}{m}+\frac{1}{n}}$\\
b. 
\begin{Schunk}
\begin{Sinput}
> f = 4 + 4 - 2
> x = c(14, 14.3, 12.2, 15.1)
> y = c(12.1, 13.6, 11.9, 11.2)
> S1 = var(x)
> S2 = var(y)
> Sp = (4 - 1)/6 * S1 + (4 - 1)/6 * S2
> m = mean(x) - mean(y)
> n = qt(0.975, f) * sqrt(Sp * ((1/4) + (1/4)))
> m - n
\end{Sinput}
\begin{Soutput}
[1] -0.2421761
\end{Soutput}
\begin{Sinput}
> m + n
\end{Sinput}
\begin{Soutput}
[1] 3.642176
\end{Soutput}
\end{Schunk}
So with 95\% confidence the difference between the means is in [-0.24,3.64].\\

c.
\begin{Schunk}
\begin{Sinput}
> t.test(x, y)
\end{Sinput}
\begin{Soutput}
	Welch Two Sample t-test

data:  x and y 
t = 2.1418, df = 5.79, p-value = 0.07762
alternative hypothesis: true difference in means is not equal to 0 
95 percent confidence interval:
 -0.2593854  3.6593854 
sample estimates:
mean of x mean of y 
     13.9      12.2 
\end{Soutput}
\end{Schunk}
So the answer here with 95 % confidence is [-0.26,3.66]. So the answer is almost the same but the interval is a bit narrower for the new test.

\item Problem 37
a.
\begin{Schunk}
\begin{Sinput}
> d = ex09.37
> t.test(d$Indoor, d$Outdoor, paired = TRUE)
\end{Sinput}
\begin{Soutput}
	Paired t-test

data:  d$Indoor and d$Outdoor 
t = -5.9509, df = 32, p-value = 1.251e-06
alternative hypothesis: true difference in means is not equal to 0 
95 percent confidence interval:
 -0.5450513 -0.2670700 
sample estimates:
mean of the differences 
             -0.4060606 
\end{Soutput}
\end{Schunk}

The null hypothesis with significance 95\% is rejected. It seems the mean for indoor is lower than the mean for outdoor.

b. The answer for is in the back of the book but it makes no sense to me. I would think that for the 34th house the CI would be the same, ie [-.55,-.27]

\item Problem 40
a.
\begin{Schunk}
\begin{Sinput}
> x = c(1928, 2549, 2825, 1924, 1628, 2175, 2114, 2621, 1843, 2541)
> y = c(2126, 2885, 2895, 1942, 1750, 2184, 2164, 2626, 2006, 2627)
> t.test(x, y, paired = TRUE, alt = "less")
\end{Sinput}
\begin{Soutput}
	Paired t-test

data:  x and y 
t = -3.2188, df = 9, p-value = 0.005255
alternative hypothesis: true difference in means is less than 0 
95 percent confidence interval:
      -Inf -45.50299 
sample estimates:
mean of the differences 
                 -105.7 
\end{Soutput}
\end{Schunk}

Yes, it does. The p-value is lower than .05.\\

b. From the t-test above, the upper confidence bound is 45.50.\\

c.
\begin{Schunk}
\begin{Sinput}
> t.test(x, y, paired = FALSE, alt = "less")
\end{Sinput}
\begin{Soutput}
	Welch Two Sample t-test

data:  x and y 
t = -0.5887, df = 17.99, p-value = 0.2817
alternative hypothesis: true difference in means is less than 0 
95 percent confidence interval:
    -Inf 205.645 
sample estimates:
mean of x mean of y 
   2214.8    2320.5 
\end{Soutput}
\end{Schunk}
No, it gives a different conclusion: one would not reject the null hypothesis.

\item Problem 50
a.
\begin{Schunk}
\begin{Sinput}
> prop.test(x = c(35, 66), n = c(80, 80), alt = "two.sided")
\end{Sinput}
\begin{Soutput}
	2-sample test for equality of proportions with continuity correction

data:  c(35, 66) out of c(80, 80) 
X-squared = 24.1651, df = 1, p-value = 8.842e-07
alternative hypothesis: two.sided 
95 percent confidence interval:
 -0.5369293 -0.2380707 
sample estimates:
prop 1 prop 2 
0.4375 0.8250 
\end{Soutput}
\end{Schunk}
The p-value is so small that I would reject the null-hypothesis at level .01.It appears the proportions differ.\\

b.
\begin{Schunk}
\begin{Sinput}
> power.prop.test(p1 = 0.5, p2 = 0.25, sig.level = 0.01, n = 80)
\end{Sinput}
\begin{Soutput}
     Two-sample comparison of proportions power calculation 

              n = 80
             p1 = 0.5
             p2 = 0.25
      sig.level = 0.01
          power = 0.762504
    alternative = two.sided

 NOTE: n is number in *each* group 
\end{Soutput}
\end{Schunk}
The probability is 76 \%.

\item Problem 62
\begin{Schunk}
\begin{Sinput}
> fu = xmp09.07$strength[1:10]
> nf = xmp09.07$strength[11:18]
> var(fu)/var(nf)
\end{Sinput}
\begin{Soutput}
[1] 1.813865
\end{Soutput}
\begin{Sinput}
> 1 - pf(1.814, 9, 7)
\end{Sinput}
\begin{Soutput}
[1] 0.2222725
\end{Soutput}
\end{Schunk}
The p-value is greater than .01 so not enough evidence to reject the null hypothesis.

\item Problem 63
\begin{Schunk}
\begin{Sinput}
> (32)^2/(54)^2
\end{Sinput}
\begin{Soutput}
[1] 0.351166
\end{Soutput}
\begin{Sinput}
> pf((32)^2/(54)^2, 22, 19)
\end{Sinput}
\begin{Soutput}
[1] 0.01008296
\end{Soutput}
\end{Schunk}
Yes, I'd reject the null hypothesis at significance level 5\% (but not at 1\%). It seems the variation in the low-dose rats is greater.

\end{enumerate}
\end{document}
